% Position Encoding

\section{Position Encoding}

Position encoding for special tokens presents unique challenges since these tokens often don't follow conventional sequential ordering rules. Special tokens may represent global context, structural boundaries, or meta-information that transcends positional constraints. This section explores strategies for effectively encoding positional information for special tokens while maintaining their semantic purpose.

\subsection{Special Token Position Assignment}

The assignment of positional information to special tokens requires careful consideration of their semantic roles and interaction patterns.

\begin{lstlisting}[language=Python, caption={Flexible position encoding for special tokens}]
# Complete implementation available at:
# https://github.com/hfgong/special-token/blob/main/code/part4/chapter10/position_encoding_flexible_position_encoding_for.py

# See the external file for the complete implementation
# File: code/part4/chapter10/position_encoding_flexible_position_encoding_for.py
# Lines: 86

class ImplementationReference:
    """Flexible position encoding for special tokens
    
    The complete implementation is available in the external code file.
    This placeholder reduces the book's verbosity while maintaining
    access to all implementation details.
    """
    pass
\end{lstlisting}

\subsection{Relative Position Encoding for Special Tokens}

Relative position encoding can be particularly effective for special tokens as it focuses on relationships rather than absolute positions.

\begin{lstlisting}[language=Python, caption={Relative position encoding with special token awareness}]
# Complete implementation available at:
# https://github.com/hfgong/special-token/blob/main/code/part4/chapter10/position_encoding_relative_position_encoding_wit.py

# See the external file for the complete implementation
# File: code/part4/chapter10/position_encoding_relative_position_encoding_wit.py
# Lines: 81

class ImplementationReference:
    """Relative position encoding with special token awareness
    
    The complete implementation is available in the external code file.
    This placeholder reduces the book's verbosity while maintaining
    access to all implementation details.
    """
    pass
\end{lstlisting}

\subsection{Learned Position Embeddings}

Learned position embeddings provide maximum flexibility for special token positioning but require careful initialization and training.

\begin{lstlisting}[language=Python, caption={Learned position embeddings with special token support}]
# Complete implementation available at:
# https://github.com/hfgong/special-token/blob/main/code/part4/chapter10/position_encoding_learned_position_embeddings_wi.py

# See the external file for the complete implementation
# File: code/part4/chapter10/position_encoding_learned_position_embeddings_wi.py
# Lines: 138

class ImplementationReference:
    """Learned position embeddings with special token support
    
    The complete implementation is available in the external code file.
    This placeholder reduces the book's verbosity while maintaining
    access to all implementation details.
    """
    pass
\end{lstlisting}

\subsection{Multi-Scale Position Encoding}

Multi-scale position encoding allows special tokens to operate at different temporal scales within the sequence.

\begin{lstlisting}[language=Python, caption={Multi-scale position encoding for hierarchical processing}]
# Complete implementation available at:
# https://github.com/hfgong/special-token/blob/main/code/part4/chapter10/position_encoding_multi-scale_position_encoding_.py

# See the external file for the complete implementation
# File: code/part4/chapter10/position_encoding_multi-scale_position_encoding_.py
# Lines: 89

class ImplementationReference:
    """Multi-scale position encoding for hierarchical processing
    
    The complete implementation is available in the external code file.
    This placeholder reduces the book's verbosity while maintaining
    access to all implementation details.
    """
    pass
\end{lstlisting}

\subsection{Best Practices for Position Encoding}

When implementing position encoding for special tokens, consider these best practices:

\begin{itemize}
\item \textbf{Semantic Alignment}: Ensure position encodings align with the semantic roles of special tokens
\item \textbf{Flexibility}: Use learnable components that can adapt to different sequence structures
\item \textbf{Scale Awareness}: Consider multi-scale encodings for tokens that operate at different temporal scales
\item \textbf{Context Sensitivity}: Allow position encodings to be influenced by sequence content when appropriate
\item \textbf{Initialization}: Carefully initialize position parameters to avoid training instabilities
\item \textbf{Regularization}: Apply appropriate regularization to prevent overfitting in position embeddings
\item \textbf{Evaluation}: Test position encoding strategies across different sequence lengths and structures
\item \textbf{Compatibility}: Ensure position encodings work well with existing pre-trained models when fine-tuning
\end{itemize}