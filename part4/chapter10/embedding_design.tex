% Embedding Design

\section{Embedding Design}

The design and initialization of special token embeddings significantly impacts model performance and training dynamics. Unlike regular token embeddings that learn from frequent occurrence in training data, special token embeddings often require careful initialization strategies and specialized training approaches to ensure they effectively capture their intended functionality.

\subsection{Initialization Strategies for Special Token Embeddings}

The initialization of special token embeddings must balance between providing useful starting points and avoiding interference with pre-existing model knowledge.

\begin{lstlisting}[language=Python, caption={Advanced initialization strategies for special token embeddings}]
# Complete implementation available at:
# https://github.com/hfgong/special-token/blob/main/code/part4/chapter10/embedding_design_advanced_initialization_strate.py

# See the external file for the complete implementation
# File: code/part4/chapter10/embedding_design_advanced_initialization_strate.py
# Lines: 106

class ImplementationReference:
    """Advanced initialization strategies for special token embeddings
    
    The complete implementation is available in the external code file.
    This placeholder reduces the book's verbosity while maintaining
    access to all implementation details.
    """
    pass
\end{lstlisting}

\subsection{Adaptive Embedding Updates}

Special token embeddings often benefit from adaptive update strategies that account for their unique roles in the model.

\begin{lstlisting}[language=Python, caption={Adaptive embedding update strategies}]
# Complete implementation available at:
# https://github.com/hfgong/special-token/blob/main/code/part4/chapter10/embedding_design_adaptive_embedding_update_stra.py

# See the external file for the complete implementation
# File: code/part4/chapter10/embedding_design_adaptive_embedding_update_stra.py
# Lines: 75

class ImplementationReference:
    """Adaptive embedding update strategies
    
    The complete implementation is available in the external code file.
    This placeholder reduces the book's verbosity while maintaining
    access to all implementation details.
    """
    pass
\end{lstlisting}

\subsection{Embedding Regularization Techniques}

Regularization helps prevent special token embeddings from diverging too far from the main embedding space while maintaining their distinctive properties.

\begin{lstlisting}[language=Python, caption={Regularization techniques for special token embeddings}]
# Complete implementation available at:
# https://github.com/hfgong/special-token/blob/main/code/part4/chapter10/embedding_design_regularization_techniques_for_.py

# See the external file for the complete implementation
# File: code/part4/chapter10/embedding_design_regularization_techniques_for_.py
# Lines: 67

class ImplementationReference:
    """Regularization techniques for special token embeddings
    
    The complete implementation is available in the external code file.
    This placeholder reduces the book's verbosity while maintaining
    access to all implementation details.
    """
    pass
\end{lstlisting}

\subsection{Dynamic Embedding Adaptation}

Special token embeddings can be dynamically adapted during training based on their usage patterns and the model's needs.

\begin{lstlisting}[language=Python, caption={Dynamic adaptation of special token embeddings}]
# Complete implementation available at:
# https://github.com/hfgong/special-token/blob/main/code/part4/chapter10/embedding_design_dynamic_adaptation_of_special_.py

# See the external file for the complete implementation
# File: code/part4/chapter10/embedding_design_dynamic_adaptation_of_special_.py
# Lines: 61

class ImplementationReference:
    """Dynamic adaptation of special token embeddings
    
    The complete implementation is available in the external code file.
    This placeholder reduces the book's verbosity while maintaining
    access to all implementation details.
    """
    pass
\end{lstlisting}

\subsection{Embedding Projection and Transformation}

Special tokens may benefit from additional projection layers that transform their embeddings based on context.

\begin{lstlisting}[language=Python, caption=Contextual projection of special token embeddings]
class SpecialTokenProjection(nn.Module):
    def __init__(self, embedding_dim=768, num_special_tokens=10):
        super().__init__()
        self.embedding_dim = embedding_dim
        self.num_special_tokens = num_special_tokens
        
        # Projection matrices for each special token
        self.projections = nn.ModuleDict({
            f'token_{i}': nn.Linear(embedding_dim, embedding_dim)
            for i in range(num_special_tokens)
        })
        
        # Context-aware gating
        self.context_gate = nn.Sequential(
            nn.Linear(embedding_dim * 2, embedding_dim),
            nn.Tanh(),
            nn.Linear(embedding_dim, embedding_dim),
            nn.Sigmoid()
        )
        
    def forward(self, embeddings, token_ids, context_embeddings=None):
        """Apply contextual projection to special token embeddings."""
        batch_size, seq_len, _ = embeddings.shape
        projected_embeddings = embeddings.clone()
        
        for i in range(self.num_special_tokens):
            # Find positions of this special token
            mask = (token_ids == i)
            
            if mask.any():
                # Get embeddings for this special token
                token_embeddings = embeddings[mask]
                
                # Apply projection
                projection = self.projections[f'token_{i}']
                projected = projection(token_embeddings)
                
                # Apply context gating if available
                if context_embeddings is not None:
                    context_for_token = context_embeddings[mask]
                    
                    # Compute gate values
                    combined = torch.cat([token_embeddings, context_for_token], dim=-1)
                    gate = self.context_gate(combined)
                    
                    # Apply gating
                    projected = gate * projected + (1 - gate) * token_embeddings
                    
                # Update embeddings
                projected_embeddings[mask] = projected
                
        return projected_embeddings
\end{lstlisting}

\subsection{Best Practices for Embedding Design}

When designing embeddings for special tokens, consider these best practices:

\begin{itemize}
\item \textbf{Initialization Strategy}: Choose initialization based on token purpose and model architecture
\item \textbf{Learning Rate Scheduling}: Use different learning rates for special vs. regular tokens
\item \textbf{Regularization}: Apply appropriate regularization to prevent overfitting
\item \textbf{Monitoring}: Track embedding evolution and usage patterns during training
\item \textbf{Adaptation}: Allow embeddings to adapt based on task requirements
\item \textbf{Evaluation}: Regularly evaluate the quality of special token representations
\item \textbf{Stability}: Ensure embeddings remain stable and don't diverge during training
\end{itemize}