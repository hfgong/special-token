% Tokenizer Modification

\section{Tokenizer Modification}

Modifying tokenizers to accommodate special tokens is a fundamental step in implementing custom transformer architectures. This process requires careful consideration of vocabulary management, encoding/decoding pipelines, and compatibility with existing preprocessing workflows.

\subsection{Extending Tokenizer Vocabularies}

The first step in tokenizer modification involves extending the vocabulary to include new special tokens while maintaining compatibility with existing tokens.
\begin{comment}
Feedback: Before the code, it's helpful to explain the core task. For example: "When you add a new special token like `<USER>`, you need to perform two key actions: 1) Add the token string to the tokenizer's vocabulary, and 2) Resize the model's token embedding matrix to create a new row for the token's vector representation. The `transformers` library provides a convenient `add_special_tokens` method that handles both of these steps, ensuring the tokenizer and model stay synchronized."
\end{comment}

\begin{lstlisting}[language=Python, caption=Safe vocabulary extension for special tokens]
class ExtendedTokenizer:
    def __init__(self, base_tokenizer, special_tokens=None):
        self.base_tokenizer = base_tokenizer
        self.special_tokens = special_tokens or {}
        self.special_token_ids = {}
        
        # Reserve token IDs for special tokens
        self._reserve_special_token_ids()
        
    def _reserve_special_token_ids(self):
        """Reserve vocabulary slots for special tokens."""
        # Get current vocabulary size
        base_vocab_size = len(self.base_tokenizer.vocab)
        
        # Assign IDs to special tokens
        for i, (token_name, token_str) in enumerate(self.special_tokens.items()):
            token_id = base_vocab_size + i
            self.special_token_ids[token_str] = token_id
            
            # Update reverse mapping
            self.base_tokenizer.ids_to_tokens[token_id] = token_str
            self.base_tokenizer.vocab[token_str] = token_id
            
        # Update vocabulary size
        self.vocab_size = base_vocab_size + len(self.special_tokens)
        
    def add_special_tokens(self, tokens_dict):
        """Dynamically add new special tokens."""
        for token_name, token_str in tokens_dict.items():
            if token_str not in self.special_token_ids:
                # Assign new ID
                new_id = self.vocab_size
                self.special_token_ids[token_str] = new_id
                self.special_tokens[token_name] = token_str
                
                # Update mappings
                self.base_tokenizer.ids_to_tokens[new_id] = token_str
                self.base_tokenizer.vocab[token_str] = new_id
                
                self.vocab_size += 1
                
        return len(tokens_dict)
\end{lstlisting}

\subsection{Encoding Pipeline Integration}

Integrating special tokens into the encoding pipeline requires careful handling of token insertion, position tracking, and segment identification.

\begin{lstlisting}[language=Python, caption=Special token-aware encoding pipeline]
class SpecialTokenEncoder:
    def __init__(self, tokenizer):
        self.tokenizer = tokenizer
        self.special_patterns = self._compile_special_patterns()
        
    def encode_with_special_tokens(self, text, add_special_tokens=True,
                                    max_length=512, task_type=None):
        """Encode text with appropriate special tokens."""
        
        # Detect and preserve special tokens in input
        preserved_tokens = self._preserve_existing_special_tokens(text)
        
        # Tokenize regular text
        if preserved_tokens:
            tokens = self._tokenize_with_preserved(text, preserved_tokens)
        else:
            tokens = self.tokenizer.tokenize(text)
            
        # Add task-specific special tokens
        if add_special_tokens:
            tokens = self._add_special_tokens(tokens, task_type)
            
        # Convert to IDs
        token_ids = self.tokenizer.convert_tokens_to_ids(tokens)
        
        # Handle truncation
        if len(token_ids) > max_length:
            token_ids = self._truncate_sequence(token_ids, max_length)
            
        # Create attention mask
        attention_mask = [1] * len(token_ids)
        
        # Create token type IDs
        token_type_ids = self._create_token_type_ids(token_ids)
        
        return {
            'input_ids': token_ids,
            'attention_mask': attention_mask,
            'token_type_ids': token_type_ids,
            'special_tokens_mask': self._create_special_tokens_mask(token_ids)
        }
        
    def _add_special_tokens(self, tokens, task_type):
        """Add appropriate special tokens based on task type."""
        if task_type == 'classification':
            tokens = [self.tokenizer.cls_token] + tokens + [self.tokenizer.sep_token]
        elif task_type == 'generation':
            tokens = [self.tokenizer.bos_token] + tokens + [self.tokenizer.eos_token]
        elif task_type == 'masked_lm':
            # Tokens already contain [MASK] tokens
            tokens = [self.tokenizer.cls_token] + tokens + [self.tokenizer.sep_token]
        elif task_type == 'dual_sequence':
            # Handle with separator tokens between sequences
            # Assumes tokens is a list of two sequences
            if isinstance(tokens[0], list):
                tokens = ([self.tokenizer.cls_token] + tokens[0] +
                         [self.tokenizer.sep_token] + tokens[1] +
                         [self.tokenizer.sep_token])
        
        return tokens
\end{lstlisting}

\subsection{Handling Special Token Collisions}

When working with pre-trained models and custom special tokens, collision handling becomes critical to avoid vocabulary conflicts.
\begin{comment}
Feedback: It's helpful to explain the problem before showing the solution. For example: "A common issue arises when a special token you want to add (e.g., `<USER>`) already exists as a regular token in the pre-trained tokenizer's vocabulary. Simply adding it again can cause ambiguity. A robust implementation should detect these collisions and either force an override (if intended) or automatically generate a non-colliding alternative (e.g., `<USER_>`)."
\end{comment}

\begin{lstlisting}[language=Python, caption=Collision detection and resolution]
class CollisionAwareTokenizer:
    def __init__(self, base_tokenizer):
        self.base_tokenizer = base_tokenizer
        self.collision_map = {}
        self.reserved_patterns = set()
        
    def register_special_token(self, token_str, force=False):
        """Register a special token with collision detection."""
        
        # Check for exact collision
        if token_str in self.base_tokenizer.vocab:
            if not force:
                # Generate alternative
                alternative = self._generate_alternative(token_str)
                self.collision_map[token_str] = alternative
                token_str = alternative
            else:
                # Override existing token
                print(f"Warning: Overriding existing token '{token_str}'")
                
        # Check for pattern collision
        if self._check_pattern_collision(token_str):
            raise ValueError(f"Token '{token_str}' conflicts with reserved pattern")
            
        # Register the token
        self._add_to_vocabulary(token_str)
        return token_str
        
    def _generate_alternative(self, token_str):
        """Generate alternative token string to avoid collision."""
        # Try adding underscores
        for i in range(1, 10):
            alternative = f"{token_str}{'_' * i}"
            if alternative not in self.base_tokenizer.vocab:
                return alternative
                
        # Try adding version number
        for i in range(1, 100):
            alternative = f"{token_str}_v{i}"
            if alternative not in self.base_tokenizer.vocab:
                return alternative
                
        raise ValueError(f"Could not find alternative for '{token_str}'")
\end{lstlisting}

\subsection{Batch Processing with Special Tokens}

Efficient batch processing requires careful handling of special tokens across sequences of different lengths, ensuring proper alignment and padding strategies.

\begin{lstlisting}[language=Python, caption=Batch processing with special token alignment]
class BatchTokenProcessor:
    def __init__(self, tokenizer, pad_to_multiple_of=8):
        self.tokenizer = tokenizer
        self.pad_to_multiple_of = pad_to_multiple_of
        
    def process_batch(self, texts, max_length=512, padding='longest'):
        """Process a batch of texts with special token handling."""
        
        # Encode all texts
        encoded_batch = []
        for text in texts:
            encoded = self.tokenizer.encode_with_special_tokens(
                text, 
                add_special_tokens=True,
                max_length=max_length
            )
            encoded_batch.append(encoded)
            
        # Determine padding length
        if padding == 'max_length':
            pad_length = max_length
        elif padding == 'longest':
            pad_length = max(len(enc['input_ids']) for enc in encoded_batch)
            # Round up to multiple if specified
            if self.pad_to_multiple_of:
                pad_length = ((pad_length + self.pad_to_multiple_of - 1) // 
                             self.pad_to_multiple_of * self.pad_to_multiple_of)
        else:
            return encoded_batch  # No padding
            
        # Apply padding
        padded_batch = self._apply_padding(encoded_batch, pad_length)
        
        # Stack into tensors
        import torch
        batch_tensors = {
            key: torch.tensor([item[key] for item in padded_batch])
            for key in padded_batch[0].keys()
        }
        
        return batch_tensors
\end{lstlisting}

\subsection{Best Practices for Tokenizer Modification}

When modifying tokenizers for special tokens, consider these best practices:

\begin{itemize}
\item \textbf{Preserve Backward Compatibility}: Always maintain compatibility with existing model checkpoints
\item \textbf{Document Special Tokens}: Maintain clear documentation of all special tokens and their purposes
\item \textbf{Test Edge Cases}: Thoroughly test handling of empty inputs, very long sequences, and special character combinations
\item \textbf{Version Control}: Implement versioning for tokenizer configurations to manage updates
\item \textbf{Performance Monitoring}: Track tokenization speed and memory usage, especially for large batches
\item \textbf{Error Handling}: Implement robust error handling for invalid token configurations
\end{itemize}
\begin{comment}
Feedback: This is a good list. To make it more actionable:
1.  **Preserve Backward Compatibility**: "When adding new special tokens, save the new tokenizer configuration to a *new* directory. Do not overwrite the original. This allows you to load both the old and new versions of the model and tokenizer, which is crucial for testing and deployment."
2.  **Version Control**: "Store your `special_tokens_map.json` and `tokenizer_config.json` files in your version control system (like Git) alongside your model code. This ensures that changes to the tokenizer are tracked and reproducible."
3.  **Test Edge Cases**: "Create a small suite of unit tests for your tokenizer that includes tricky cases: strings that look like your special tokens, empty strings, strings with mixed special and normal tokens, etc. This can prevent many common bugs."
\end{comment}

