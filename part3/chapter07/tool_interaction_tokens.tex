\section{Tool Interaction and API Tokens}

The integration of external tools and APIs into language models through specialized tokens represents a pivotal evolution in AI capabilities. Tool interaction tokens transform language models from isolated reasoning engines into orchestrators of complex, multi-system workflows, enabling them to access real-time information, perform calculations, manipulate data, and interact with the physical world through connected systems.

\subsection{The Tool-Use Paradigm Shift}

Traditional language models operate within the confines of their training data and parametric knowledge. Tool interaction tokens break these boundaries, enabling models to:

\begin{itemize}
\item Access real-time information beyond their training cutoff
\item Perform precise calculations without approximation
\item Interact with external databases and knowledge systems
\item Execute code and system commands
\item Orchestrate complex multi-tool workflows
\end{itemize}

This paradigm shift transforms AI from a passive question-answering system to an active agent capable of taking actions in digital and physical environments.

\subsection{Fundamental Tool Token Architecture}

\subsubsection{Basic Tool Invocation Tokens}

The simplest form of tool interaction involves tokens that delimit tool calls:

\begin{lstlisting}[language=Python, caption=Basic tool invocation token system]
class ToolInteractionTokenizer:
    def __init__(self, base_tokenizer, available_tools):
        self.base_tokenizer = base_tokenizer
        self.available_tools = available_tools
        self.tool_tokens = {
            'tool_call_start': '[TOOL_CALL]',
            'tool_call_end': '[/TOOL_CALL]',
            'tool_name': '[TOOL]',
            'parameters': '[PARAMS]',
            'result_start': '[RESULT]',
            'result_end': '[/RESULT]',
            'error': '[ERROR]'
        }
        
    def format_tool_call(self, tool_name, parameters):
        """Format a tool call with appropriate tokens"""
        if tool_name not in self.available_tools:
            raise ValueError(f"Tool {tool_name} not available")
        
        formatted = (
            f"{self.tool_tokens['tool_call_start']}\n"
            f"{self.tool_tokens['tool_name']} {tool_name}\n"
            f"{self.tool_tokens['parameters']} {json.dumps(parameters)}\n"
            f"{self.tool_tokens['tool_call_end']}"
        )
        
        return formatted
    
    def parse_model_output_for_tools(self, model_output):
        """Parse model output to extract tool calls"""
        import re
        import json
        
        tool_calls = []
        
        # Pattern to match tool calls
        pattern = (
            f"{re.escape(self.tool_tokens['tool_call_start'])}"
            f".*?"
            f"{re.escape(self.tool_tokens['tool_call_end'])}"
        )
        
        matches = re.findall(pattern, model_output, re.DOTALL)
        
        for match in matches:
            # Extract tool name
            tool_pattern = f"{re.escape(self.tool_tokens['tool_name'])}\\s+(\\w+)"
            tool_match = re.search(tool_pattern, match)
            tool_name = tool_match.group(1) if tool_match else None
            
            # Extract parameters
            params_pattern = f"{re.escape(self.tool_tokens['parameters'])}\\s+(\\{{.*?\\}})"
            params_match = re.search(params_pattern, match, re.DOTALL)
            
            if params_match:
                try:
                    parameters = json.loads(params_match.group(1))
                except json.JSONDecodeError:
                    parameters = {}
            else:
                parameters = {}
            
            if tool_name:
                tool_calls.append({
                    'tool': tool_name,
                    'parameters': parameters
                })
        
        return tool_calls
    
    def execute_tool_call(self, tool_name, parameters):
        """Execute a tool call and return formatted result"""
        try:
            # Get the tool function
            tool_func = self.available_tools[tool_name]
            
            # Execute the tool
            result = tool_func(**parameters)
            
            # Format successful result
            formatted_result = (
                f"{self.tool_tokens['result_start']}\n"
                f"{json.dumps(result, indent=2)}\n"
                f"{self.tool_tokens['result_end']}"
            )
            
        except Exception as e:
            # Format error result
            formatted_result = (
                f"{self.tool_tokens['error']}\n"
                f"Tool execution failed: {str(e)}\n"
                f"{self.tool_tokens['result_end']}"
            )
        
        return formatted_result
    
    def create_tool_augmented_prompt(self, user_query, tool_results=None):
        """Create prompt with tool usage context"""
        prompt = f"User Query: {user_query}\n\n"
        
        # Add available tools description
        prompt += "Available tools:\n"
        for tool_name, tool_func in self.available_tools.items():
            if hasattr(tool_func, '__doc__'):
                prompt += f"- {tool_name}: {tool_func.__doc__}\n"
        
        # Add previous tool results if any
        if tool_results:
            prompt += "\nPrevious tool results:\n"
            for result in tool_results:
                prompt += result + "\n"
        
        prompt += "\nResponse (use tools as needed):\n"
        
        return prompt
\end{lstlisting}

\subsection{Structured API Call Tokens}

More sophisticated systems use structured tokens for complex API interactions:

\begin{lstlisting}[language=Python, caption=Structured API call tokenization]
class APICallTokenizer:
    def __init__(self, base_tokenizer):
        self.base_tokenizer = base_tokenizer
        self.api_tokens = {
            'api_start': '[API_CALL]',
            'api_end': '[/API_CALL]',
            'endpoint': '[ENDPOINT]',
            'method': '[METHOD]',
            'headers': '[HEADERS]',
            'body': '[BODY]',
            'response': '[RESPONSE]',
            'status': '[STATUS]',
            'auth': '[AUTH]'
        }
        
    def format_api_call(self, endpoint, method='GET', headers=None, body=None, auth=None):
        """Format an API call with structured tokens"""
        formatted = [self.api_tokens['api_start']]
        
        # Add endpoint
        formatted.append(f"{self.api_tokens['endpoint']} {endpoint}")
        
        # Add HTTP method
        formatted.append(f"{self.api_tokens['method']} {method}")
        
        # Add headers if present
        if headers:
            formatted.append(f"{self.api_tokens['headers']} {json.dumps(headers)}")
        
        # Add authentication if present
        if auth:
            formatted.append(f"{self.api_tokens['auth']} [REDACTED]")
        
        # Add body for POST/PUT requests
        if body and method in ['POST', 'PUT', 'PATCH']:
            formatted.append(f"{self.api_tokens['body']} {json.dumps(body)}")
        
        formatted.append(self.api_tokens['api_end'])
        
        return '\n'.join(formatted)
    
    def parse_api_response(self, response, status_code):
        """Parse and format API response with tokens"""
        formatted = [
            self.api_tokens['response'],
            f"{self.api_tokens['status']} {status_code}"
        ]
        
        if 200 <= status_code < 300:
            # Successful response
            formatted.append(json.dumps(response, indent=2))
        else:
            # Error response
            formatted.append(f"[ERROR] API call failed with status {status_code}")
            if response:
                formatted.append(json.dumps(response, indent=2))
        
        return '\n'.join(formatted)
    
    def create_api_chain(self, api_calls):
        """Create a chain of dependent API calls"""
        chain = []
        results = {}
        
        for i, call in enumerate(api_calls):
            # Substitute variables from previous results
            if 'depends_on' in call:
                for dep_idx, dep_field in call['depends_on'].items():
                    if dep_idx in results:
                        # Replace placeholder with actual value
                        placeholder = f"${{result_{dep_idx}.{dep_field}}}"
                        actual_value = results[dep_idx].get(dep_field, '')
                        
                        # Update endpoint or body with actual value
                        if 'endpoint' in call:
                            call['endpoint'] = call['endpoint'].replace(
                                placeholder, str(actual_value)
                            )
                        if 'body' in call:
                            call['body'] = json.loads(
                                json.dumps(call['body']).replace(
                                    placeholder, str(actual_value)
                                )
                            )
            
            # Format the API call
            formatted_call = self.format_api_call(**call)
            chain.append(formatted_call)
            
            # Simulate execution and store result
            # In real implementation, this would actually call the API
            results[i] = {'simulated': 'result'}
        
        return chain
\end{lstlisting}

\subsection{Function Calling Tokens}

Function calling tokens enable models to invoke specific functions with typed parameters:

\begin{lstlisting}[language=Python, caption=Function calling token system]
class FunctionCallingTokenizer:
    def __init__(self, base_tokenizer, function_registry):
        self.base_tokenizer = base_tokenizer
        self.function_registry = function_registry
        self.func_tokens = {
            'call': '[FUNC_CALL]',
            'name': '[FUNC_NAME]',
            'args': '[ARGS]',
            'kwargs': '[KWARGS]',
            'return': '[RETURN]',
            'type': '[TYPE]',
            'async': '[ASYNC]',
            'await': '[AWAIT]'
        }
        
    def format_function_call(self, func_name, *args, **kwargs):
        """Format a function call with type information"""
        if func_name not in self.function_registry:
            raise ValueError(f"Function {func_name} not registered")
        
        func_info = self.function_registry[func_name]
        
        formatted = [
            self.func_tokens['call'],
            f"{self.func_tokens['name']} {func_name}"
        ]
        
        # Add type information
        if 'return_type' in func_info:
            formatted.append(f"{self.func_tokens['type']} {func_info['return_type']}")
        
        # Add positional arguments
        if args:
            formatted.append(f"{self.func_tokens['args']} {list(args)}")
        
        # Add keyword arguments
        if kwargs:
            formatted.append(f"{self.func_tokens['kwargs']} {kwargs}")
        
        # Mark if async
        if func_info.get('is_async', False):
            formatted.append(self.func_tokens['async'])
        
        return '\n'.join(formatted)
    
    def validate_function_call(self, func_name, args, kwargs):
        """Validate function call against signature"""
        import inspect
        
        if func_name not in self.function_registry:
            return False, "Function not found"
        
        func = self.function_registry[func_name]['function']
        sig = inspect.signature(func)
        
        try:
            # Check if arguments match signature
            bound = sig.bind(*args, **kwargs)
            bound.apply_defaults()
            return True, "Valid call"
        except TypeError as e:
            return False, str(e)
    
    def execute_function_safely(self, func_name, args, kwargs):
        """Execute function with safety checks"""
        # Validate first
        is_valid, message = self.validate_function_call(func_name, args, kwargs)
        
        if not is_valid:
            return {
                'error': True,
                'message': message
            }
        
        func_info = self.function_registry[func_name]
        func = func_info['function']
        
        # Apply safety restrictions
        if func_info.get('requires_confirmation', False):
            # In real implementation, would ask for user confirmation
            pass
        
        if func_info.get('rate_limited', False):
            # Check rate limits
            pass
        
        try:
            # Execute function
            if func_info.get('is_async', False):
                # Handle async execution
                import asyncio
                result = asyncio.run(func(*args, **kwargs))
            else:
                result = func(*args, **kwargs)
            
            return {
                'error': False,
                'result': result,
                'type': type(result).__name__
            }
            
        except Exception as e:
            return {
                'error': True,
                'message': str(e),
                'type': 'exception'
            }
    
    def create_function_chain(self, chain_spec):
        """Create a chain of function calls with data flow"""
        results = []
        context = {}
        
        for step in chain_spec:
            func_name = step['function']
            
            # Resolve arguments from context
            resolved_args = []
            for arg in step.get('args', []):
                if isinstance(arg, str) and arg.startswith('$'):
                    # Reference to previous result
                    var_name = arg[1:]
                    if var_name in context:
                        resolved_args.append(context[var_name])
                    else:
                        resolved_args.append(arg)
                else:
                    resolved_args.append(arg)
            
            # Execute function
            result = self.execute_function_safely(
                func_name,
                resolved_args,
                step.get('kwargs', {})
            )
            
            # Store result in context
            if 'store_as' in step:
                context[step['store_as']] = result.get('result')
            
            results.append({
                'step': step,
                'result': result
            })
        
        return results
\end{lstlisting}

\subsection{Database and Query Tokens}

Specialized tokens for database interactions:

\begin{lstlisting}[language=Python, caption=Database query tokens]
class DatabaseQueryTokenizer:
    def __init__(self, base_tokenizer):
        self.base_tokenizer = base_tokenizer
        self.db_tokens = {
            'query_start': '[DB_QUERY]',
            'query_end': '[/DB_QUERY]',
            'sql': '[SQL]',
            'nosql': '[NOSQL]',
            'collection': '[COLLECTION]',
            'table': '[TABLE]',
            'result_set': '[RESULT_SET]',
            'row_count': '[ROW_COUNT]',
            'transaction': '[TRANSACTION]',
            'commit': '[COMMIT]',
            'rollback': '[ROLLBACK]'
        }
        
    def format_sql_query(self, query, params=None, transaction=False):
        """Format SQL query with safety tokens"""
        formatted = []
        
        if transaction:
            formatted.append(self.db_tokens['transaction'])
        
        formatted.extend([
            self.db_tokens['query_start'],
            f"{self.db_tokens['sql']} {query}"
        ])
        
        if params:
            # Use parameterized queries for safety
            formatted.append(f"[PARAMS] {params}")
        
        formatted.append(self.db_tokens['query_end'])
        
        return '\n'.join(formatted)
    
    def format_nosql_query(self, collection, operation, filter_doc=None, update_doc=None):
        """Format NoSQL query with tokens"""
        formatted = [
            self.db_tokens['query_start'],
            f"{self.db_tokens['nosql']}",
            f"{self.db_tokens['collection']} {collection}",
            f"[OPERATION] {operation}"
        ]
        
        if filter_doc:
            formatted.append(f"[FILTER] {json.dumps(filter_doc)}")
        
        if update_doc and operation in ['update', 'replace']:
            formatted.append(f"[UPDATE] {json.dumps(update_doc)}")
        
        formatted.append(self.db_tokens['query_end'])
        
        return '\n'.join(formatted)
    
    def parse_query_result(self, result, query_type='sql'):
        """Parse and format query results"""
        formatted = [self.db_tokens['result_set']]
        
        if query_type == 'sql':
            # Format as table
            if result and len(result) > 0:
                # Get column names
                columns = list(result[0].keys())
                formatted.append(f"[COLUMNS] {columns}")
                
                # Add row count
                formatted.append(f"{self.db_tokens['row_count']} {len(result)}")
                
                # Add first few rows
                for i, row in enumerate(result[:5]):
                    formatted.append(f"[ROW_{i}] {row}")
                
                if len(result) > 5:
                    formatted.append(f"... and {len(result) - 5} more rows")
            else:
                formatted.append("[EMPTY_RESULT]")
                
        elif query_type == 'nosql':
            # Format as documents
            formatted.append(f"[DOCUMENT_COUNT] {len(result)}")
            for i, doc in enumerate(result[:3]):
                formatted.append(f"[DOC_{i}] {json.dumps(doc, indent=2)}")
            
            if len(result) > 3:
                formatted.append(f"... and {len(result) - 3} more documents")
        
        return '\n'.join(formatted)
    
    def create_transaction_block(self, queries):
        """Create a transaction block with multiple queries"""
        transaction = [
            self.db_tokens['transaction'],
            "[BEGIN]"
        ]
        
        for i, query in enumerate(queries):
            transaction.append(f"[STEP_{i}]")
            transaction.append(self.format_sql_query(query['sql'], query.get('params')))
            
            # Add validation check
            if 'validate' in query:
                transaction.append(f"[VALIDATE] {query['validate']}")
        
        transaction.extend([
            "[END]",
            self.db_tokens['commit']
        ])
        
        return '\n'.join(transaction)
\end{lstlisting}

\subsection{File System and IO Tokens}

Tokens for file system operations:

\begin{lstlisting}[language=Python, caption=File system operation tokens]
class FileSystemTokenizer:
    def __init__(self, base_tokenizer, sandbox_path="/tmp/sandbox"):
        self.base_tokenizer = base_tokenizer
        self.sandbox_path = sandbox_path
        self.fs_tokens = {
            'read': '[FS_READ]',
            'write': '[FS_WRITE]',
            'append': '[FS_APPEND]',
            'delete': '[FS_DELETE]',
            'mkdir': '[FS_MKDIR]',
            'list': '[FS_LIST]',
            'path': '[PATH]',
            'content': '[CONTENT]',
            'permissions': '[PERMS]',
            'size': '[SIZE]',
            'modified': '[MODIFIED]'
        }
        
    def format_file_operation(self, operation, path, content=None):
        """Format file operation with safety checks"""
        import os
        
        # Ensure path is within sandbox
        full_path = os.path.join(self.sandbox_path, path)
        if not full_path.startswith(self.sandbox_path):
            raise ValueError("Path escapes sandbox")
        
        formatted = [
            self.fs_tokens[operation],
            f"{self.fs_tokens['path']} {path}"
        ]
        
        if operation in ['write', 'append'] and content:
            # Limit content size
            max_size = 1024 * 1024  # 1MB
            if len(content) > max_size:
                content = content[:max_size] + "\n[TRUNCATED]"
            
            formatted.append(f"{self.fs_tokens['content']}")
            formatted.append(content)
        
        return '\n'.join(formatted)
    
    def format_directory_listing(self, path, recursive=False):
        """Format directory listing with metadata"""
        import os
        from datetime import datetime
        
        full_path = os.path.join(self.sandbox_path, path)
        
        listing = [
            self.fs_tokens['list'],
            f"{self.fs_tokens['path']} {path}"
        ]
        
        try:
            for item in os.listdir(full_path):
                item_path = os.path.join(full_path, item)
                stat = os.stat(item_path)
                
                item_info = {
                    'name': item,
                    'type': 'DIR' if os.path.isdir(item_path) else 'FILE',
                    'size': stat.st_size,
                    'modified': datetime.fromtimestamp(stat.st_mtime).isoformat()
                }
                
                listing.append(f"[ITEM] {json.dumps(item_info)}")
                
                if recursive and os.path.isdir(item_path):
                    # Recursively list subdirectories
                    sublisting = self.format_directory_listing(
                        os.path.join(path, item),
                        recursive=True
                    )
                    listing.append(sublisting)
                    
        except PermissionError:
            listing.append("[ERROR] Permission denied")
        except FileNotFoundError:
            listing.append("[ERROR] Path not found")
        
        return '\n'.join(listing)
\end{lstlisting}

\subsection{Web Scraping and Browser Automation Tokens}

Tokens for web interaction and data extraction:

\begin{lstlisting}[language=Python, caption=Web scraping and browser automation tokens]
class WebScrapingTokenizer:
    def __init__(self, base_tokenizer):
        self.base_tokenizer = base_tokenizer
        self.web_tokens = {
            'navigate': '[WEB_NAVIGATE]',
            'click': '[WEB_CLICK]',
            'type': '[WEB_TYPE]',
            'select': '[WEB_SELECT]',
            'extract': '[WEB_EXTRACT]',
            'screenshot': '[WEB_SCREENSHOT]',
            'wait': '[WEB_WAIT]',
            'selector': '[SELECTOR]',
            'xpath': '[XPATH]',
            'url': '[URL]',
            'element': '[ELEMENT]'
        }
        
    def format_navigation(self, url, wait_for=None):
        """Format web navigation action"""
        formatted = [
            self.web_tokens['navigate'],
            f"{self.web_tokens['url']} {url}"
        ]
        
        if wait_for:
            formatted.append(f"{self.web_tokens['wait']} {wait_for}")
        
        return '\n'.join(formatted)
    
    def format_interaction(self, action, selector, value=None):
        """Format web interaction action"""
        formatted = [
            self.web_tokens[action],
            f"{self.web_tokens['selector']} {selector}"
        ]
        
        if value:
            formatted.append(f"[VALUE] {value}")
        
        return '\n'.join(formatted)
    
    def format_extraction(self, selectors, extract_type='text'):
        """Format data extraction from web page"""
        formatted = [
            self.web_tokens['extract'],
            f"[EXTRACT_TYPE] {extract_type}"
        ]
        
        for name, selector in selectors.items():
            formatted.append(f"[FIELD] {name} -> {selector}")
        
        return '\n'.join(formatted)
    
    def create_scraping_workflow(self, steps):
        """Create a complete web scraping workflow"""
        workflow = []
        
        for i, step in enumerate(steps):
            workflow.append(f"[STEP_{i}]")
            
            if step['action'] == 'navigate':
                workflow.append(self.format_navigation(
                    step['url'],
                    step.get('wait_for')
                ))
            elif step['action'] in ['click', 'type', 'select']:
                workflow.append(self.format_interaction(
                    step['action'],
                    step['selector'],
                    step.get('value')
                ))
            elif step['action'] == 'extract':
                workflow.append(self.format_extraction(
                    step['selectors'],
                    step.get('type', 'text')
                ))
            elif step['action'] == 'screenshot':
                workflow.append(f"{self.web_tokens['screenshot']} {step.get('filename', 'screenshot.png')}")
            
            # Add validation if specified
            if 'validate' in step:
                workflow.append(f"[VALIDATE] {step['validate']}")
        
        return '\n'.join(workflow)
\end{lstlisting}

\subsection{Multi-Tool Orchestration}

Complex tasks often require orchestrating multiple tools:

\begin{lstlisting}[language=Python, caption=Multi-tool orchestration system]
class MultiToolOrchestrator:
    def __init__(self, tokenizers):
        self.tokenizers = tokenizers  # Dict of tool-specific tokenizers
        self.orchestration_tokens = {
            'workflow_start': '[WORKFLOW]',
            'workflow_end': '[/WORKFLOW]',
            'parallel': '[PARALLEL]',
            'sequential': '[SEQUENTIAL]',
            'conditional': '[IF]',
            'loop': '[LOOP]',
            'variable': '[VAR]',
            'checkpoint': '[CHECKPOINT]'
        }
        
    def create_workflow(self, workflow_spec):
        """Create a multi-tool workflow from specification"""
        workflow = [self.orchestration_tokens['workflow_start']]
        
        for task in workflow_spec['tasks']:
            if task['type'] == 'parallel':
                workflow.append(self.format_parallel_tasks(task['subtasks']))
            elif task['type'] == 'sequential':
                workflow.append(self.format_sequential_tasks(task['subtasks']))
            elif task['type'] == 'conditional':
                workflow.append(self.format_conditional_task(task))
            elif task['type'] == 'loop':
                workflow.append(self.format_loop_task(task))
            else:
                workflow.append(self.format_single_task(task))
        
        workflow.append(self.orchestration_tokens['workflow_end'])
        
        return '\n'.join(workflow)
    
    def format_parallel_tasks(self, tasks):
        """Format tasks to run in parallel"""
        formatted = [self.orchestration_tokens['parallel']]
        
        for i, task in enumerate(tasks):
            formatted.append(f"[THREAD_{i}]")
            formatted.append(self.format_single_task(task))
        
        formatted.append("[WAIT_ALL]")
        
        return '\n'.join(formatted)
    
    def format_sequential_tasks(self, tasks):
        """Format tasks to run sequentially"""
        formatted = [self.orchestration_tokens['sequential']]
        
        for i, task in enumerate(tasks):
            formatted.append(f"[STEP_{i}]")
            formatted.append(self.format_single_task(task))
            
            # Add checkpoint after each task
            if task.get('checkpoint', False):
                formatted.append(f"{self.orchestration_tokens['checkpoint']} step_{i}")
        
        return '\n'.join(formatted)
    
    def format_conditional_task(self, task):
        """Format conditional task execution"""
        formatted = [
            f"{self.orchestration_tokens['conditional']} {task['condition']}",
            "[THEN]",
            self.format_single_task(task['then_task'])
        ]
        
        if 'else_task' in task:
            formatted.extend([
                "[ELSE]",
                self.format_single_task(task['else_task'])
            ])
        
        formatted.append("[END_IF]")
        
        return '\n'.join(formatted)
    
    def format_loop_task(self, task):
        """Format loop task execution"""
        formatted = [
            f"{self.orchestration_tokens['loop']} {task['iteration_type']}"
        ]
        
        if task['iteration_type'] == 'for':
            formatted.append(f"[RANGE] {task['range']}")
        elif task['iteration_type'] == 'while':
            formatted.append(f"[CONDITION] {task['condition']}")
        elif task['iteration_type'] == 'foreach':
            formatted.append(f"[ITEMS] {task['items']}")
        
        formatted.append("[DO]")
        formatted.append(self.format_single_task(task['body']))
        formatted.append("[END_LOOP]")
        
        return '\n'.join(formatted)
    
    def format_single_task(self, task):
        """Format a single tool task"""
        tool_type = task['tool']
        
        if tool_type in self.tokenizers:
            tokenizer = self.tokenizers[tool_type]
            
            # Use appropriate tokenizer based on tool type
            if tool_type == 'api':
                return tokenizer.format_api_call(**task['params'])
            elif tool_type == 'database':
                return tokenizer.format_sql_query(**task['params'])
            elif tool_type == 'file':
                return tokenizer.format_file_operation(**task['params'])
            elif tool_type == 'web':
                return tokenizer.format_navigation(**task['params'])
            else:
                return f"[UNKNOWN_TOOL] {tool_type}"
        
        return f"[TOOL] {tool_type} {json.dumps(task.get('params', {}))}"
\end{lstlisting}

\subsection{Safety and Security Considerations}

Tool interaction tokens require careful security measures:

\begin{enumerate}
\item \textbf{Sandboxing}: Execute tools in isolated environments
\item \textbf{Permission Systems}: Implement granular permission controls
\item \textbf{Rate Limiting}: Prevent abuse through rate limits
\item \textbf{Input Validation}: Validate all parameters before execution
\item \textbf{Audit Logging}: Log all tool interactions for security review
\item \textbf{Confirmation Requirements}: Require user confirmation for sensitive operations
\end{enumerate}

\subsection{Best Practices for Tool Token Design}

Effective tool interaction systems follow key principles:

\begin{itemize}
\item \textbf{Clear Delimitation}: Unambiguous token boundaries for parsing
\item \textbf{Type Safety}: Include type information for parameters and returns
\item \textbf{Error Handling}: Comprehensive error tokens and recovery mechanisms
\item \textbf{Idempotency}: Design for safe retry of failed operations
\item \textbf{Observability}: Include tokens for monitoring and debugging
\item \textbf{Composability}: Enable complex workflows through token composition
\end{itemize}

Tool interaction and API tokens transform language models into powerful orchestrators of digital systems. By providing structured interfaces to external tools, these tokens enable AI systems to take real actions, access current information, and solve complex problems that require interaction with multiple systems. As the ecosystem of available tools continues to expand, the sophistication and importance of tool interaction tokens will only grow, making them a cornerstone of practical AI applications.