% Pretraining Strategies for Special Tokens

\section{Pretraining Strategies}

Pretraining forms the foundation for effective special token development, establishing the basic representations and functional capabilities that will be refined during subsequent training phases. Unlike standard language model pretraining that focuses primarily on next-token prediction, pretraining with special tokens requires specialized strategies that explicitly train their intended functions. This section presents targeted approaches for pretraining special tokens effectively.
\begin{comment}
Feedback: Before diving into the specifics, it's helpful to frame the core challenge. For example: "The main challenge in pretraining special tokens is that they are often 'meta-level' instructions that are not naturally present in raw text. A standard pretraining objective like next-token prediction doesn't explicitly teach a [CLS] token how to summarize, or a [SEP] token how to separate concepts. Therefore, specialized pretraining strategies are needed to create learning signals that directly encourage these desired functions."
\end{comment}

\subsection{Curriculum Design for Special Token Development}

The design of pretraining curricula significantly impacts the quality of special token function development. Effective curricula provide appropriate learning signals while maintaining training stability and efficiency.

\subsubsection{Progressive Complexity Curricula}

Progressive complexity curricula introduce special token functions gradually, starting with simple tasks and progressively increasing complexity as training proceeds.
\begin{comment}
Feedback: A concrete example would make this concept much clearer. For instance: "A progressive curriculum for a model with [CLS] and [SEP] tokens might look like this:
1.  **Phase 1 (Language Modeling)**: Train the model on a standard masked language modeling (MLM) objective *without* the Next Sentence Prediction (NSP) task. This allows the content embeddings to stabilize and learn basic language.
2.  **Phase 2 (Introduce NSP)**: After the MLM loss has started to converge, introduce the NSP task, which specifically trains the [CLS] and [SEP] tokens to understand relationships between sentences.
This two-phase approach is often more stable than trying to learn everything simultaneously from scratch."
\end{comment}

\begin{lstlisting}[language=Python, caption={Progressive curriculum framework for special token pretraining}]
# Complete implementation available at:
# https://github.com/hfgong/special-token/blob/main/code/part3/chapter09/pretraining_strategies_progressive_curriculum_framewo.py

# See the external file for the complete implementation
# File: code/part3/chapter09/pretraining_strategies_progressive_curriculum_framewo.py
# Lines: 380

class ImplementationReference:
    """Progressive curriculum framework for special token pretraining
    
    The complete implementation is available in the external code file.
    This placeholder reduces the book's verbosity while maintaining
    access to all implementation details.
    """
    pass
\end{lstlisting}

\subsection{Specialized Pretraining Objectives}

Standard language modeling objectives may not provide optimal learning signals for special token development. Specialized objectives can enhance the development of specific special token functions.

\subsubsection{Function-Specific Loss Components}

Different special tokens require different types of learning signals to develop their intended functions effectively.

\subsubsection{Multi-Task Pretraining}

Multi-task pretraining can provide diverse learning signals that encourage the development of robust and generalizable special token representations.

\subsection{Special Token-Specific Training Objectives}

Beyond standard language modeling, special tokens benefit from targeted training objectives that directly encourage their intended functions.

\subsubsection{Function-Specific Auxiliary Tasks}

Auxiliary tasks can be designed to specifically train special token functions. For example, sentence classification tasks for \cls{} tokens or boundary prediction tasks for \sep{} tokens.

\subsubsection{Self-Supervised Special Token Tasks}

Self-supervised tasks can be created from existing text to provide learning signals for special tokens without requiring additional labeled data.