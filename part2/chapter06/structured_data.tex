% Structured Data Processing Section

\section{Structured Data Processing}

Structured data processing represents a critical domain where transformer architectures must navigate complex relationships between entities, schemas, and hierarchical data organizations. Unlike unstructured text or visual data, structured data processing requires tokens that understand database schemas, query languages, data relationships, and transformation pipelines while maintaining referential integrity and supporting complex analytical operations.

The integration of specialized tokens in structured data processing enables AI systems to assist with database design, query optimization, data migration, ETL pipeline development, and automated data analysis workflows while ensuring data quality and consistency across diverse data sources and formats.

\subsection{Schema-Aware Tokens}

Structured data processing requires specialized tokens that understand database schemas, relationships, and constraints.

\subsubsection{Database Schema Tokens}

Database operations require tokens that can represent tables, columns, relationships, and constraints.

The complete implementation is provided in the external code file \texttt{../../code/part2/chapter06/schemaaware\_database\_tokenization\_system.py}. Key components include:

\begin{lstlisting}[language=Python, caption=Core structure (see external file for complete implementation)]
# See ../../code/part2/chapter06/schemaaware_database_tokenization_system.py for the complete implementation
# This shows only the main class structure
class DatabaseSchemaTokenizer:
    # ... (complete implementation in external file)
    pass
\end{lstlisting}
\subsubsection{Data Transformation Tokens}

ETL and data transformation pipelines require specialized tokens for operations and data flow.

\begin{lstlisting}[language=Python, caption=Data transformation and ETL tokenization]
class DataTransformationTokenizer:
    def __init__(self, base_tokenizer):
        self.base_tokenizer = base_tokenizer
        
        # ETL operation tokens
        self.etl_tokens = {
            'EXTRACT': '<EXTRACT>',
            'TRANSFORM': '<TRANSFORM>',
            'LOAD': '<LOAD>',
            'FILTER': '<FILTER>',
            'MAP': '<MAP>',
            'REDUCE': '<REDUCE>',
            'AGGREGATE': '<AGGREGATE>',
            'PIVOT': '<PIVOT>',
            'UNPIVOT': '<UNPIVOT>',
            'UNION': '<UNION>',
            'INTERSECT': '<INTERSECT>',
        }
        
        # Data flow tokens
        self.flow_tokens = {
            'SOURCE': '<SOURCE>',
            'SINK': '<SINK>',
            'PIPELINE_START': '<PIPELINE_START>',
            'PIPELINE_END': '<PIPELINE_END>',
            'STEP_START': '<STEP_START>',
            'STEP_END': '<STEP_END>',
            'DEPENDENCY': '<DEPENDENCY>',
            'PARALLEL': '<PARALLEL>',
        }
        
        # Data quality tokens
        self.quality_tokens = {
            'VALIDATE': '<VALIDATE>',
            'CLEAN': '<CLEAN>',
            'DEDUPE': '<DEDUPE>',
            'STANDARDIZE': '<STANDARDIZE>',
            'ENRICH': '<ENRICH>',
            'QUALITY_CHECK': '<QUALITY_CHECK>',
        }
    
    def tokenize_pipeline(self, pipeline_definition):
        """Tokenize data transformation pipeline."""
        tokens = []
        tokens.append(self.flow_tokens['PIPELINE_START'])
        
        for step in pipeline_definition['steps']:
            tokens.append(self.flow_tokens['STEP_START'])
            
            # Add operation token
            if step['operation'] in self.etl_tokens:
                tokens.append(self.etl_tokens[step['operation']])
            
            # Add data quality operations
            if 'quality_checks' in step:
                for check in step['quality_checks']:
                    if check in self.quality_tokens:
                        tokens.append(self.quality_tokens[check])
            
            # Tokenize step configuration
            step_tokens = self.base_tokenizer.tokenize(str(step['config']))
            tokens.extend(step_tokens)
            
            tokens.append(self.flow_tokens['STEP_END'])
        
        tokens.append(self.flow_tokens['PIPELINE_END'])
        return tokens

class DataPipelineTransformer(nn.Module):
    def __init__(self, vocab_size, embed_dim=768):
        super().__init__()
        
        self.structured_transformer = StructuredDataTransformer(vocab_size, embed_dim)
        
        # Pipeline-specific embeddings
        self.operation_embeddings = nn.Embedding(20, embed_dim)  # ETL operations
        self.flow_embeddings = nn.Embedding(15, embed_dim)  # Data flow patterns
        
        # Pipeline optimization network
        self.pipeline_optimizer = PipelineOptimizationNetwork(embed_dim)
        
        # Data quality analyzer
        self.quality_analyzer = DataQualityNetwork(embed_dim)
        
    def forward(self, input_ids, pipeline_structure=None):
        # Process through structured transformer
        output = self.structured_transformer(input_ids)
        
        # Add pipeline-specific information
        if pipeline_structure is not None:
            pipeline_embeds = self.encode_pipeline_structure(pipeline_structure)
            output = output + pipeline_embeds
        
        return output
    
    def encode_pipeline_structure(self, pipeline_structure):
        """Encode pipeline structure information."""
        operation_embeds = self.operation_embeddings(
            pipeline_structure['operations']
        )
        flow_embeds = self.flow_embeddings(pipeline_structure['flow_pattern'])
        
        return operation_embeds + flow_embeds
    
    def optimize_pipeline(self, pipeline_tokens):
        """Optimize data transformation pipeline."""
        return self.pipeline_optimizer(pipeline_tokens)
    
    def analyze_quality(self, data_tokens):
        """Analyze data quality issues."""
        return self.quality_analyzer(data_tokens)

class PipelineOptimizationNetwork(nn.Module):
    def __init__(self, embed_dim):
        super().__init__()
        
        self.optimization_network = nn.Sequential(
            nn.Linear(embed_dim, embed_dim // 2),
            nn.ReLU(),
            nn.Linear(embed_dim // 2, embed_dim // 4),
            nn.ReLU(),
            nn.Linear(embed_dim // 4, 10)  # Optimization suggestions
        )
        
    def forward(self, pipeline_embed):
        return self.optimization_network(pipeline_embed)

class DataQualityNetwork(nn.Module):
    def __init__(self, embed_dim):
        super().__init__()
        
        self.quality_network = nn.Sequential(
            nn.Linear(embed_dim, embed_dim // 2),
            nn.ReLU(),
            nn.Linear(embed_dim // 2, 20)  # Quality metrics
        )
        
    def forward(self, data_embed):
        return self.quality_network(data_embed)
\end{lstlisting}

\subsection{Query Generation and Optimization}

\subsubsection{Natural Language to SQL Translation}

\begin{lstlisting}[language=Python, caption=Natural language to SQL generation system]
class NL2SQLTransformer(nn.Module):
    def __init__(self, vocab_size, embed_dim=768):
        super().__init__()
        
        self.data_transformer = DataPipelineTransformer(vocab_size, embed_dim)
        
        # Natural language encoder
        self.nl_encoder = nn.TransformerEncoder(
            nn.TransformerEncoderLayer(embed_dim, nhead=12, batch_first=True),
            num_layers=6
        )
        
        # SQL decoder
        self.sql_decoder = nn.TransformerDecoder(
            nn.TransformerDecoderLayer(embed_dim, nhead=12, batch_first=True),
            num_layers=6
        )
        
        # Schema-aware attention
        self.schema_attention = nn.MultiheadAttention(
            embed_dim, num_heads=8, batch_first=True
        )
        
        # Query optimization head
        self.query_optimizer = nn.Sequential(
            nn.Linear(embed_dim, embed_dim // 2),
            nn.ReLU(),
            nn.Linear(embed_dim // 2, vocab_size)
        )
        
    def forward(self, nl_query, schema_context, target_sql=None):
        # Encode natural language query
        nl_encoded = self.nl_encoder(nl_query)
        
        # Encode schema context
        schema_encoded = self.data_transformer(schema_context)
        
        # Schema-aware attention
        query_context, _ = self.schema_attention(
            nl_encoded, schema_encoded, schema_encoded
        )
        
        if target_sql is not None:
            # Training mode: generate SQL with teacher forcing
            sql_output = self.sql_decoder(target_sql, query_context)
        else:
            # Inference mode: generate SQL autoregressively
            sql_output = self.generate_sql(query_context)
        
        # Optimize generated query
        optimized_sql = self.query_optimizer(sql_output)
        
        return optimized_sql
    
    def generate_sql(self, query_context, max_length=200):
        """Generate SQL query autoregressively."""
        batch_size = query_context.size(0)
        device = query_context.device
        
        # Start with special token
        generated = torch.zeros(batch_size, 1, dtype=torch.long, device=device)
        
        for i in range(max_length):
            # Decode next token
            output = self.sql_decoder(generated, query_context)
            next_token = torch.argmax(output[:, -1, :], dim=-1, keepdim=True)
            generated = torch.cat([generated, next_token], dim=1)
            
            # Check for end token
            if torch.all(next_token == 2):  # Assuming 2 is end token
                break
        
        return generated
\end{lstlisting}

\subsection{Best Practices for Structured Data Processing}

Implementing effective structured data processing tokens requires several key considerations:

\begin{enumerate}
\item \textbf{Schema Awareness}: Maintain understanding of database structures and relationships
\item \textbf{Query Optimization}: Support efficient query generation and optimization
\item \textbf{Data Quality}: Integrate data validation and quality checking mechanisms
\item \textbf{Referential Integrity}: Ensure consistency across related data elements
\item \textbf{Scalability}: Design for large-scale data processing requirements
\item \textbf{Security}: Implement appropriate access controls and data privacy measures
\item \textbf{Interoperability}: Support multiple data formats and database systems
\item \textbf{Pipeline Management}: Enable complex ETL and data transformation workflows
\end{enumerate}
The complete implementation is provided in the external code file \texttt{../../code/part2/chapter06/data\_transformation\_and\_etl\_tokenization.py}. Key components include:

\begin{lstlisting}[language=Python, caption=Core structure (see external file for complete implementation)]
# See ../../code/part2/chapter06/data_transformation_and_etl_tokenization.py for the complete implementation
# This shows only the main class structure
Structured data processing tokens enable AI systems to work effectively with databases, data warehouses, and complex data processing pipelines, supporting automated database design, query optimization, and intelligent data transformation while maintaining data integrity and performance requirements.
    # ... (complete implementation in external file)
    pass
\end{lstlisting}