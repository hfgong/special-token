% Code Generation Models Section

\section{Code Generation Models}

Code generation models represent one of the most successful applications of transformer architectures to domain-specific tasks, enabling AI systems to understand, generate, and manipulate source code across multiple programming languages. The unique challenges of code processing—including strict syntactic requirements, complex semantic relationships, and the need for executable output—have driven the development of specialized tokens that capture the structural and semantic properties of programming languages.

Unlike natural language, code has precise syntactic rules, hierarchical structures, and execution semantics that must be preserved for the output to be functional. This necessitates special tokens that understand programming constructs, maintain syntactic correctness, and enable sophisticated code understanding and generation capabilities.

\subsection{Programming Language Special Tokens}

Effective code generation requires specialized tokens that capture the unique aspects of programming languages.

\subsubsection{Language Switching Tokens}

Multi-language code generation requires tokens that can signal transitions between different programming languages within the same context.

\begin{lstlisting}[language=Python, caption=Language switching tokens for multi-language code generation]
class MultiLanguageCodeTransformer(nn.Module):
    def __init__(self, vocab_size, embed_dim=768, num_languages=10):
        super().__init__()
        
        # Base transformer
        self.transformer = nn.TransformerEncoder(
            nn.TransformerEncoderLayer(
                d_model=embed_dim,
                nhead=12,
                batch_first=True
            ),
            num_layers=12
        )
        
        # Language-specific embeddings
        self.language_embeddings = nn.Embedding(num_languages, embed_dim)
        self.token_embeddings = nn.Embedding(vocab_size, embed_dim)
        
        # Language switching tokens
        self.language_switch_tokens = nn.ParameterDict({
            'python': nn.Parameter(torch.randn(1, embed_dim)),
            'javascript': nn.Parameter(torch.randn(1, embed_dim)),
            'java': nn.Parameter(torch.randn(1, embed_dim)),
            'cpp': nn.Parameter(torch.randn(1, embed_dim)),
            'rust': nn.Parameter(torch.randn(1, embed_dim)),
        })
        
        # Language-specific code heads
        self.language_heads = nn.ModuleDict({
            lang: nn.Linear(embed_dim, vocab_size) 
            for lang in self.language_switch_tokens.keys()
        })
        
    def forward(self, input_ids, language_ids):
        # Token embeddings
        token_embeds = self.token_embeddings(input_ids)
        
        # Language embeddings
        lang_embeds = self.language_embeddings(language_ids)
        
        # Combine embeddings
        combined_embeds = token_embeds + lang_embeds
        
        # Add language switch tokens at appropriate positions
        enhanced_embeds = self.add_language_switches(combined_embeds, language_ids)
        
        # Transformer processing
        output = self.transformer(enhanced_embeds)
        
        return output
    
    def add_language_switches(self, embeddings, language_ids):
        """Add language switch tokens at language transition points."""
        batch_size, seq_len, embed_dim = embeddings.shape
        
        # Detect language transitions
        transitions = (language_ids[:, 1:] != language_ids[:, :-1])
        
        enhanced_embeddings = []
        for b in range(batch_size):
            sequence = [embeddings[b, 0]]  # Start with first token
            
            for i in range(1, seq_len):
                if transitions[b, i-1]:  # Language transition detected
                    new_lang_id = language_ids[b, i].item()
                    lang_name = self.get_language_name(new_lang_id)
                    
                    if lang_name in self.language_switch_tokens:
                        switch_token = self.language_switch_tokens[lang_name]
                        sequence.append(switch_token.squeeze(0))
                
                sequence.append(embeddings[b, i])
            
            # Pad to original length
            while len(sequence) < seq_len:
                sequence.append(torch.zeros(embed_dim, device=embeddings.device))
            
            enhanced_embeddings.append(torch.stack(sequence[:seq_len]))
        
        return torch.stack(enhanced_embeddings)
\end{lstlisting}

\subsubsection{Indentation and Structure Tokens}

Code structure is heavily dependent on indentation and hierarchical organization.

\begin{lstlisting}[language=Python, caption=Structure-aware code tokenization]
class StructuralCodeTokenizer:
    def __init__(self, base_tokenizer):
        self.base_tokenizer = base_tokenizer
        
        # Structural special tokens
        self.special_tokens = {
            'INDENT': '<INDENT>',
            'DEDENT': '<DEDENT>',
            'NEWLINE': '<NEWLINE>',
            'FUNC_DEF': '<FUNC_DEF>',
            'CLASS_DEF': '<CLASS_DEF>',
            'VAR_DEF': '<VAR_DEF>',
            'IMPORT': '<IMPORT>',
        }
        
    def tokenize_with_structure(self, code_text):
        """Tokenize code while preserving structural information."""
        lines = code_text.split('\n')
        tokens = []
        indent_stack = [0]
        
        for line in lines:
            stripped_line = line.lstrip()
            if not stripped_line:
                tokens.append(self.special_tokens['NEWLINE'])
                continue
            
            current_indent = len(line) - len(stripped_line)
            
            # Handle indentation changes
            if current_indent > indent_stack[-1]:
                indent_stack.append(current_indent)
                tokens.append(self.special_tokens['INDENT'])
            elif current_indent < indent_stack[-1]:
                while indent_stack and current_indent < indent_stack[-1]:
                    indent_stack.pop()
                    tokens.append(self.special_tokens['DEDENT'])
            
            # Add structural markers
            if stripped_line.startswith('def '):
                tokens.append(self.special_tokens['FUNC_DEF'])
            elif stripped_line.startswith('class '):
                tokens.append(self.special_tokens['CLASS_DEF'])
            elif stripped_line.startswith('import '):
                tokens.append(self.special_tokens['IMPORT'])
            
            # Tokenize actual content
            line_tokens = self.base_tokenizer.tokenize(stripped_line)
            tokens.extend(line_tokens)
            tokens.append(self.special_tokens['NEWLINE'])
        
        return tokens
\end{lstlisting}

\subsection{Code Completion Applications}

\begin{lstlisting}[language=Python, caption=Advanced code completion system]
class AdvancedCodeCompletion(nn.Module):
    def __init__(self, vocab_size, embed_dim=768):
        super().__init__()
        
        self.code_model = MultiLanguageCodeTransformer(vocab_size, embed_dim)
        
        # Context encoders
        self.file_context_encoder = nn.TransformerEncoder(
            nn.TransformerEncoderLayer(embed_dim, nhead=8, batch_first=True),
            num_layers=3
        )
        
        # Special tokens for completion
        self.completion_tokens = nn.ParameterDict({
            'cursor': nn.Parameter(torch.randn(1, embed_dim)),
            'context_start': nn.Parameter(torch.randn(1, embed_dim)),
        })
        
        # Completion scoring
        self.completion_scorer = nn.Linear(embed_dim, vocab_size)
    
    def forward(self, current_code, cursor_position, file_context=None):
        # Encode current code
        code_repr = self.code_model(current_code, torch.zeros_like(current_code))
        
        # Add cursor position information
        cursor_token = self.completion_tokens['cursor']
        # Insert cursor token at position (simplified)
        
        # Generate completion scores
        completion_scores = self.completion_scorer(code_repr)
        
        return completion_scores[:, cursor_position, :]
    
    def generate_completions(self, code_text, cursor_pos, num_completions=5):
        """Generate code completion suggestions."""
        # Tokenize input
        tokens = self.tokenize_code(code_text)
        
        # Get completion scores
        scores = self.forward(tokens, cursor_pos)
        
        # Return top completions
        top_scores, top_indices = torch.topk(scores, num_completions)
        return self.decode_completions(top_indices)
\end{lstlisting}

\subsection{Best Practices for Code Generation}

Implementing effective code generation requires several key considerations:

\begin{enumerate}
\item \textbf{Syntax Preservation}: Maintain syntactic correctness in generated code
\item \textbf{Context Awareness}: Consider broader code context and project structure
\item \textbf{Language Specificity}: Adapt to programming language paradigms
\item \textbf{Error Handling}: Provide robust error recovery mechanisms
\item \textbf{Performance}: Optimize for real-time code assistance
\end{enumerate}

Code generation models with specialized tokens have revolutionized software development by enabling intelligent code completion, automated refactoring, and sophisticated code understanding capabilities.