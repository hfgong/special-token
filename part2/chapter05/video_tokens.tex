% Video Frame Tokens Section

\section{Video Frame Tokens}

Video frame tokens represent the temporal extension of image tokens, enabling transformer architectures to process sequential visual information across time. Unlike static image tokens that capture spatial relationships within a single frame, video tokens must encode both spatial and temporal dependencies, making them fundamental for video understanding, generation, and multimodal video-text tasks.

The challenge of video representation lies in balancing the rich temporal information with computational efficiency, as videos contain orders of magnitude more data than static images. Video frame tokens serve as compressed temporal representations that maintain essential motion dynamics while remaining compatible with transformer architectures.

\subsection{Temporal Video Representation}

Video tokens must capture the temporal evolution of visual scenes while maintaining computational tractability.

\begin{definition}[Video Frame Token]
A Video Frame token is a learnable special token that represents temporal visual content within a video sequence. It encodes both spatial features within frames and temporal relationships across frames, enabling video understanding and generation tasks.
\end{definition}

\begin{lstlisting}[language=Python, caption={Video frame token architecture}]
# Complete implementation available at:
# https://github.com/hfgong/special-token/blob/main/code/part2/chapter05/video_tokens_video_frame_token_architecture.py

# See the external file for the complete implementation
# File: code/part2/chapter05/video_tokens_video_frame_token_architecture.py
# Lines: 78

class ImplementationReference:
    """Video frame token architecture
    
    The complete implementation is available in the external code file.
    This placeholder reduces the book's verbosity while maintaining
    access to all implementation details.
    """
    pass
\end{lstlisting}

\subsection{Video-Text Applications}

Video tokens enable sophisticated video-language understanding tasks.

\subsubsection{Video Captioning}

\begin{lstlisting}[language=Python, caption=Video captioning with temporal tokens]
class VideoCaptioningModel(nn.Module):
    def __init__(self, vocab_size, embed_dim=768):
        super().__init__()
        
        self.video_text_model = VideoTextTransformer(vocab_size, embed_dim)
        self.max_caption_length = 50
        
    def generate_caption(self, video_frames):
        batch_size = video_frames.shape[0]
        device = video_frames.device
        
        # Start with BOS token
        caption = torch.full((batch_size, 1), 1, device=device, dtype=torch.long)
        
        for _ in range(self.max_caption_length):
            # Generate next token
            logits = self.video_text_model(caption, video_frames)
            next_token_logits = logits[:, -1, :]
            next_tokens = torch.argmax(next_token_logits, dim=-1, keepdim=True)
            
            caption = torch.cat([caption, next_tokens], dim=1)
            
            # Check for EOS
            if (next_tokens == 2).all():  # EOS token
                break
        
        return caption
\end{lstlisting}

\subsection{Best Practices for Video Tokens}

\begin{enumerate}
\item \textbf{Frame Sampling}: Use appropriate temporal sampling strategies (uniform, adaptive)
\item \textbf{Motion Modeling}: Incorporate explicit motion features when necessary
\item \textbf{Memory Efficiency}: Balance temporal resolution with computational constraints
\item \textbf{Multi-Scale Processing}: Handle videos of different lengths and frame rates
\item \textbf{Temporal Alignment}: Synchronize video tokens with audio and text when available
\end{enumerate}

Video frame tokens extend the power of multimodal transformers to temporal visual understanding, enabling applications in video captioning, temporal action recognition, and video-text retrieval.
